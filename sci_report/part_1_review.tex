\chapter{Обзор предмета исследования}\label{ch:ch1}
\todo{2 pages}

Данная глава посвящена формулированию предмета исследования, определения его составляющих.
Выявляется прикладная проблематика и связанный с ней аппарат вычислительной физики и методов анализа данных. Проводится обзор научных и прикладных разработок.

\section{Численное моделирование разработки нефтегазовых месторождений}

Данный раздел описывает современное моделирование подземных течений углеводородов в неоднородных пористых средах, индуцированных системой добывающих и нагнетательных скважин.
Описываются отличительные особенности соответствующей области, например, от вычислительной аэро- и гидродинамики.
Современное нефтегазовое моделирование требует высокого пространственно-временного разрешения, учёта множества нелинейных процессов, связанных со свойствами нетрадиционными месторождений.
Кроме того, моделирование реальных месторождений неизбежно сопровождается значительной неопределённостью его параметров, недостатком данных и их точности.
Данные особенности обуславливают потребность решения характерных для области обратных задач.

\subsection{Обратные задачи нефтегазового моделирования}
К характерным обратным задачам относятся, в том числе, задача определения оптимального расположения источников (нефтегазовых скважин) \todo{\cite{}} и задача определения пространственных свойств месторождения по истории наблюдений параметров разработки месторождений \todo{\cite{}}.
Как правило, автоматизированные алгоритмы решения этих задач предполагают многократное итерационное решение прямых задач с изменением тех или иных параметров задачи: начальные условия, интенсивности и положения источников, распределение геологических свойств. В то время как эффективное решение таких обратных задач представляет собой отдельное направление исследований \todo{\cite{Elizarev2020,Elizarev_2021}}, ускоренное решение прямых задач в рамках таких алгоритмов вносит решающий вклад в скорость решения обратной задачи. Такое ускорение как правило достигается путём построения приближенных прокси-моделей.

\section{Прокси-моделирование}

Существует
