\chapter{Обзор литературы и прикладных разработок}\label{ch:ch1}
\todo{2 pages}

% К таким задачам относятся, в том числе, задача определения оптимального расположения источников (нефтегазовых скважин) \todo{\cite{}} и задача определения пространственных свойств месторождения по истории наблюдений параметров разработки месторождений \todo{\cite{}}.
% Как правило, автоматизированные алгоритмы решения этих задач предполагают многократное итерационное решение прямых задач с изменением тех или иных параметров задачи: начальные условия, интенсивности и положения источников, распределение геологических свойств. В то время как эффективное решение таких обратных задач представляет собой отдельное направление исследований \todo{\cite{Elizarev2020,Elizarev_2021}}, ускоренное решение прямой задачи
