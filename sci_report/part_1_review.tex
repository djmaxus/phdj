\chapter{Обзор предмета исследования}\label{ch:ch1}
\todo{citiations}
Данная глава посвящена формулированию предмета исследования, определения его составляющих.
Выявляется прикладная проблематика и связанный с ней аппарат вычислительной физики и методов анализа данных. Проводится обзор научных и прикладных разработок.

\section{Численное моделирование разработки нефтегазовых месторождений}

Данный раздел описывает современное моделирование подземных течений углеводородов в неоднородных пористых средах, индуцированных системой добывающих и нагнетательных скважин.
Описываются отличительные особенности соответствующей области, например, от вычислительной аэро- и гидродинамики.
Современное нефтегазовое моделирование требует высокого пространственно-временного разрешения, учёта множества нелинейных процессов, связанных со свойствами нетрадиционными месторождений.
Кроме того, моделирование реальных месторождений неизбежно сопровождается значительной неопределённостью его параметров, недостатком данных и их точности.
Данные особенности обуславливают потребность решения характерных для области обратных задач.

\subsection{Обратные задачи нефтегазового моделирования}
К характерным обратным задачам относятся, в том числе, задача определения оптимального расположения источников (нефтегазовых скважин) \todo{\cite{}} и задача определения пространственных свойств месторождения по истории наблюдений параметров разработки месторождений \todo{\cite{}}.
Как правило, автоматизированные алгоритмы решения этих задач предполагают многократное итерационное решение прямых задач с изменением тех или иных параметров задачи: начальные условия, интенсивности и положения источников, распределение геологических свойств. В то время как эффективное решение таких обратных задач представляет собой отдельное направление исследований \todo{\cite{Elizarev2020,Elizarev_2021}}, ускоренное решение прямых задач в рамках таких алгоритмов вносит решающий вклад в скорость решения обратной задачи. Такое ускорение как правило достигается путём построения приближенных прокси-моделей.

\section{Прокси-моделирование}

Существует целый набор альтернативных подходов к прокси-моделированию, классифицируемых по характеру применяемых аппроксимаций и явному присутствию исходных определяющих уравнений в прокси-модели.

В качестве базового подхода возможно рассматривать использование более грубых сеток как по пространству, так и по времени. Определяющие уравнения сохраняются неизменными, а аппроксимация выражается в сниженном разрешении численного решения и невыраженности эффектов более мелкого масштаба.

В качестве прокси-моделей другого вида могут выступать альтернативные дискретизации исходных дифференциальных уравнений, в том числе с упрощением физической модели в рамках тех или иных допущений. Например, таким подходом является построение линий тока, вдоль которых численно решаются одномерные уравнения, интерполируемые далее на всю расчётную область.

Ещё одной альтернативой являются так называемые суррогатные модели, широко применяемые на практике.
В отличие от предыдущих категорий, суррогатные модели строятся на эмпирических экспериментальных или синтетических данных с помощью методов анализа данных и машинного обучения в частности.
Исходные определяющие уравнения при этом могут содержаться только в целевой функции алгоритма обучения модели, что может затруднять интерпретацию суррогатной модели и оценку её применимости за пределами обучающей выборки.
Однако развиваются современные подходы, позволяющие в той или иной степени включить определяющие уравнения в архитектуру суррогатной модели, повышая качество прокси-моделирования при сохранении высокой скорости расчётов.

Рассматриваемая же в рамках данной работы альтернатива может быть классифицирована как гибридная.
Эмпирическое низкоразмерное моделирование (в англоязычной терминологии известное как Data-Driven Reduced Order Modeling) использует эмпирические данные для выделения доминирующих паттернов, определяющих репрезентативные низкоразмерные подпространства для поиска численного решения исходных определяющих уравнений или эмпирической интерполяции нелинейных функций высокой размерности.

Данный класс методов широко применяется, например, в аэро- и гидродинамике, а применение в нефтегазовом моделировании активно исследуется.
Тем не менее, эмпирическое снижение размерности не представлено широко в прикладных разработках, включая индустриальное программное обеспечение.
Это может быть связано в том числе со свойствами пространства параметров, в рамках которого необходимо получать репрезентативные низкоразмерные модели, обеспечивая при этом совокупное ускорение расчётов.
Так, к варьируемым параметрам может относиться положение точечных источников, что приближает задачу построение низкоразмерной репрезентации к исследованию нелинейных фундаментальных решений.

В данной работе систематизировано описание построения и применения эмпирического снижения размерности к прямому моделированию пластовых течений углеводородов, предложены сценарии извлечения и применения данных в рамках итерационных алгоритмов решения обратных задач. Рассмотрен вопрос построения репрезентативных низкоразмерных моделях при вариациях пространственного расположения точечных источников.
