
{\actuality}
\textbf{Актуальность} данной работы обеспечивается потребностью развития методов ускоренного численного моделирования физических процессов с помощью данных.
Рассматриваемые методы способны обеспечить ускорение такого моделирования без снижения пространственно-временного разрешения и пренебрежения теми или иными нелинейными членами определяющих уравнений.
Такие методы, широко применяемые в исследовательских и прикладных задачах, например, в области вычислительной гидродинамики, должны быть адаптированы и расширены для применимости в характерных постановках обратных задач нефтегазового моделирования {\cite{Elizarev2020,Elizarev_2021}}.
Для достижения этих целей необходима разработка методологии обеспечения обобщающей способности алгоритмов снижения размерности на пространствах параметров высокой размерности.
Кроме того, такая методология должна быть воплощена в конкретных алгоритмах.

{\progress}
\textbf{В ходе работы} аналитически и практически изучены современные методы низкоразмерного представления нелинейных динамических систем, включая методы исследования моделей по данным численных и синтетических экспериментов: методы линеаризованного представления, моделирование систем с управлением, восстановление определяющих уравнений по данным. Набор таких методов реализован для решения модельных и прикладных задач нефтегазового моделирования. Исследована задача смещения источниковых членов в модели нефтегазового месторождения в рамках понятий нелинейного фундаментального решения, предложены и проверены гипотезы о свойствах решения со смещенными источниками относительно положений в обучающих выборках.

{\aim} данной работы является разработка методологии, метрик качества и эффективных методов снижения размерности нелинейных динамических моделей вычислительной физики с управляющими воздействиями и неоднородными стационарными полями свойств среды.
В качестве методов снижения размерности рассматриваются методы анализа данных как альтернатива иным методам упрощения вычислений: снижение пространственно-временного разрешения численной модели, упрощение физической модели и т.п.
Одним из приложений разрабатываемой методологии являются обратные задачи моделирования подземных течений, пространство поиска которых может содержать координаты точечных источников и пространственное распределение фильтрационно-емкостных свойств.

Для~достижения поставленной цели необходимо было решить следующие {\tasks}:
\begin{enumerate}[beginpenalty=10000] % https://tex.stackexchange.com/a/476052/104425
  \item Проанализировать существующие исследования и разработки в контексте предмета данной работы.
  \item Определить основную проблематику, границы применимости существующих подходов.
  \item Сформулировать методологию и критерии оценки качества снижения размерности.
  \item Предложить алгоритмы, расширяющие границы применимости снижения размерности.
  \item Провести сравнительные численные эксперименты, проанализировать качество и производительность численного моделирования
  \item Сформулировать подходы к разработке дизайна прикладного программного обеспечения
\end{enumerate}

{\novelty}:
\begin{enumerate}[beginpenalty=10000] % https://tex.stackexchange.com/a/476052/104425
  \item Было выполнено оригинальное систематизирующее аналитическое исследование применимости научно-прикладных разработок в области данной работы применительно к моделированию пластовых течений в процессе разработки месторождений углеводородов.
  \item Разработан принцип повышения обобщающей способности линейного снижения размерности динамических моделей с точечными управляющими воздействиями и неоднородностями на основе инвариантных преобразований координат, элементов теории управления и теории оператора Купмана.
  \item В рамках предложенного принципа разработан набор методов снижения размерности для систем с точечным управлением и неоднородностями.
  \item Предложен с формальным обоснованием критерий сходимости моделей со сниженной размерностью, предназначенный для сопоставления со сходимостью соответствующей полноразмерной модели.
\end{enumerate}

{\influence} полученных результатов обеспечивается приближенностью рассматриваемых постановок задач к практическим задачам в сфере моделирования месторождений углеводородов. Учтены практические ограничения во времени, доступных вычислительных мощностях и данных. В представляемой работе рассматривается вопрос синергетической эффективности совокупности алгоритмов на примере открытой библиотеки для прямого и обратного нефтегазового моделирования, для которой было разработано расширение, воплощающее теоретические результаты данной работы.

{\methods}
В работе были использованы методы и подходы из ряда областей. Для описания математической модели разработки месторождения применялись методы численного решения систем нелинейных дифференциальных уравнений, методы разложения и обращения матриц (включая методы решения систем линейных уравнений). В контексте базовых алгоритмов анализа данных применялись метод главных компонент (правильного ортогонального разложения), метод косых проекций, низкоразмерные аппроксимации тензоров, линеаризация систем в терминах оператора Купмана, разложение на правильные динамические моды. При исследовании свойств нелинейных фундаментальных решений применялся аппарат инвариантного преобразования координат в дифференциальных уравнениях, методы разбиения на подобласти, методы интерполяции сеточных данных. Применительно к прикладным аспектам программной реализации вычислительных методов использовались понятия вычислительной сложности, профилирования потребления ресурсов программ и т. п. Экспериментальная проверка качества разрабатываемых методов осуществлялась с помощью синтетических численных экспериментов, псевдо-случайных чисел и кросс-верификации с результатами других авторов.

{\defpositions}
\begin{enumerate}[beginpenalty=10000] % https://tex.stackexchange.com/a/476052/104425
%   \item {это тезисы, которые никем ранее не были выдвинуты}
%   \item {Это своеобразные результаты научной деятельности, выводы, которые показывают, насколько полезно проведенное исследование и какова его ценность}
%   \item {https://disszakaz.ru/services/kandidatskaya-dissertatsiya/osnovnye-polozheniya-vynosimye-na-zashchitu-dissertatsii/}
%   \item {Def-positions.pdf}
  \item Иерархический метод построения низкоразмерных моделей многофазной нелинейной фильтрации
  \item Аппроксимация нелинейных функций высокой размерности в сочетании с линейными операциями. Предварительное вычисление низкоразмерных линейных операторов.
  \item Критерий сходимости методов решения систем нелинейных уравнений для сопоставления сходимости полноразмерных и низкоразмерных моделей.
  \item Метод задания подмножества обучающей выборки главных компонент с помощью триангуляции пространства параметров
  \item Метод предобуславливания нелинейной задачи с помощью извлечения динамических мод из данных
  \item Низкоразмерное представление нелинейного фундаментального решения методом приближенных кусочных инвариантных преобразований координат и разделения корреляций в данных по направлениям.
\end{enumerate}

{\reliability} полученных результатов обеспечивается выбранной методологией исследования.
Разработанные подходы и алгоритмы тестировались с применением элементов теории численного эксперимента, статистической проверки гипотез, и принципа фальсифицируемости.
Ряд численных экспериментов проводились с использованием псевдо-случайных входных данных.
Кроме того, для верификации полученных результатов использовались данные, модели и алгоритмы других авторов.
Таким образом, воспроизводимость результатов оценивалась непосредственно в рамках исследования.
Структура и содержание данной работы обеспечивают возможность независимого воспроизведения численных моделей, разработанных методов и полученных результатов.

{\probation}
Основные результаты работы докладывались~на международной конференции \flqq{ECMOR}\frqq~\cite{Elizarev2022,Elizarev2020}, 61-й Всероссийской конференции МФТИ~\cite{Elizarev_MIPT_conference}, в рамках международной конференции и выставки общества \flqq{SPE}\frqq~\cite{Elizarev_2019}, а также на семинарах кафедры моделирования и технологий разработки нефтяных месторождений МФТИ.

{\contribution} Автор выполнил исследование научно-технических достижений в области данной работы, интерпретировал и систематизировал их совокупность. Автор поставил и провёл необходимые теоретические и экспериментальные исследования, сформулировал проблематику и формальные постановки решаемых задач, подобрал инструменты программирования и программное обеспечение, проанализировал и оформил полученные результаты в виде научных публикаций и презентаций.

\ifnumequal{\value{bibliosel}}{0}
{%%% Встроенная реализация с загрузкой файла через движок bibtex8. (При желании, внутри можно использовать обычные ссылки, наподобие `\cite{vakbib1,vakbib2}`).
    {\publications} Основные результаты по теме диссертации изложены
    в~XX~печатных изданиях,
    X из которых изданы в журналах, рекомендованных ВАК,
    X "--- в тезисах докладов.
}%
{%%% Реализация пакетом biblatex через движок biber
    \begin{refsection}[bl-author, bl-registered]
        % Это refsection=1.
        % Процитированные здесь работы:
        %  * подсчитываются, для автоматического составления фразы "Основные результаты ..."
        %  * попадают в авторскую библиографию, при usefootcite==0 и стиле `\insertbiblioauthor` или `\insertbiblioauthorgrouped`
        %  * нумеруются там в зависимости от порядка команд `\printbibliography` в этом разделе.
        %  * при использовании `\insertbiblioauthorgrouped`, порядок команд `\printbibliography` в нём должен быть тем же (см. biblio/biblatex.tex)
        %
        % Невидимый библиографический список для подсчёта количества публикаций:
        \printbibliography[heading=nobibheading, section=1, env=countauthorvak,          keyword=biblioauthorvak]%
        \printbibliography[heading=nobibheading, section=1, env=countauthorwos,          keyword=biblioauthorwos]%
        \printbibliography[heading=nobibheading, section=1, env=countauthorscopus,       keyword=biblioauthorscopus]%
        \printbibliography[heading=nobibheading, section=1, env=countauthorconf,         keyword=biblioauthorconf]%
        \printbibliography[heading=nobibheading, section=1, env=countauthorother,        keyword=biblioauthorother]%
        \printbibliography[heading=nobibheading, section=1, env=countregistered,         keyword=biblioregistered]%
        \printbibliography[heading=nobibheading, section=1, env=countauthorpatent,       keyword=biblioauthorpatent]%
        \printbibliography[heading=nobibheading, section=1, env=countauthorprogram,      keyword=biblioauthorprogram]%
        \printbibliography[heading=nobibheading, section=1, env=countauthor,             keyword=biblioauthor]%
        \printbibliography[heading=nobibheading, section=1, env=countauthorvakscopuswos, filter=vakscopuswos]%
        \printbibliography[heading=nobibheading, section=1, env=countauthorscopuswos,    filter=scopuswos]%
        %
        \nocite{*}%
        %
        {\publications} Основные результаты по теме диссертации изложены в~\arabic{citeauthor}~печатных изданиях,
        \arabic{citeauthorvak} из которых изданы в журналах, рекомендованных ВАК\sloppy%
        \ifnum \value{citeauthorscopuswos}>0%
            , \arabic{citeauthorscopuswos} "--- в~периодических научных журналах, индексируемых Web of~Science и Scopus\sloppy%
        \fi%
        \ifnum \value{citeauthorconf}>0%
            , \arabic{citeauthorconf} "--- в~тезисах докладов.
        \else%
            .
        \fi%
        \ifnum \value{citeregistered}=1%
            \ifnum \value{citeauthorpatent}=1%
                Зарегистрирован \arabic{citeauthorpatent} патент.
            \fi%
            \ifnum \value{citeauthorprogram}=1%
                Зарегистрирована \arabic{citeauthorprogram} программа для ЭВМ.
            \fi%
        \fi%
        \ifnum \value{citeregistered}>1%
            Зарегистрированы\ %
            \ifnum \value{citeauthorpatent}>0%
            \formbytotal{citeauthorpatent}{патент}{}{а}{}\sloppy%
            \ifnum \value{citeauthorprogram}=0 . \else \ и~\fi%
            \fi%
            \ifnum \value{citeauthorprogram}>0%
            \formbytotal{citeauthorprogram}{программ}{а}{ы}{} для ЭВМ.
            \fi%
        \fi%
        % К публикациям, в которых излагаются основные научные результаты диссертации на соискание учёной
        % степени, в рецензируемых изданиях приравниваются патенты на изобретения, патенты (свидетельства) на
        % полезную модель, патенты на промышленный образец, патенты на селекционные достижения, свидетельства
        % на программу для электронных вычислительных машин, базу данных, топологию интегральных микросхем,
        % зарегистрированные в установленном порядке.(в ред. Постановления Правительства РФ от 21.04.2016 N 335)
    \end{refsection}%
    \begin{refsection}[bl-author, bl-registered]
        % Это refsection=2.
        % Процитированные здесь работы:
        %  * попадают в авторскую библиографию, при usefootcite==0 и стиле `\insertbiblioauthorimportant`.
        %  * ни на что не влияют в противном случае
        \nocite{vakbib2}%vak
        \nocite{patbib1}%patent
        \nocite{progbib1}%program
        \nocite{bib1}%other
        \nocite{confbib1}%conf
    \end{refsection}%
        %
        % Всё, что вне этих двух refsection, это refsection=0,
        %  * для диссертации - это нормальные ссылки, попадающие в обычную библиографию
        %  * для автореферата:
        %     * при usefootcite==0, ссылка корректно сработает только для источника из `external.bib`. Для своих работ --- напечатает "[0]" (и даже Warning не вылезет).
        %     * при usefootcite==1, ссылка сработает нормально. В авторской библиографии будут только процитированные в refsection=0 работы.
}
