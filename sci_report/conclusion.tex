\chapter*{Заключение}                       % Заголовок
\addcontentsline{toc}{chapter}{Заключение}  % Добавляем его в оглавление
%% Согласно ГОСТ Р 7.0.11-2011:
%% 5.3.3 В заключении диссертации излагают итоги выполненного исследования, рекомендации, перспективы дальнейшей разработки темы.
%% 9.2.3 В заключении автореферата диссертации излагают итоги данного исследования, рекомендации и перспективы дальнейшей разработки темы.
%% Поэтому имеет смысл сделать эту часть общей и загрузить из одного файла в автореферат и в диссертацию:

В заключении сформулированы основные результаты исследования.
\begin{enumerate}
    \item Характерные обратные задачи нефтегазового моделирования требуют особого подхода к быстрому и качественному прокси-моделированию пластовых течений с точечными источниками в неоднородных средах. Эмпирические низкоразмерные модели могут быть использованы в качестве-прокси моделей, однако требуют развития существующих методов и систематизации их построения в распространённых на практике сценариях.
    \item Систематически сформулированы и реализованы основные методы эмпирического низкоразмерного моделирования. Для низкоразмерных моделей предложена метрика сходимости, позволяющая переиспользовать параметры требуемой точности, задаваемые для аналогичных моделей полной размерности.
    \item Сформулирован и протестирован подход подбора и формирования комбинированных главных компонент по нескольким точкам обучающей выборки в пространстве параметров.
    \item Для эмпирической интерполяции нелинейных функций в том числе сформулирован подход предварительного вычисления низкоразмерных линейных операторов.
    \item При наличии смещения положения источника в пространстве параметров разработан подход приближенных кусочных инвариантных преобразований. Проанализирована его применимость в двумерном пространстве и при неоднородных изменениях пространственных полей.
\end{enumerate}

% И какая-нибудь заключающая фраза.
% Резюмировать ценности и значимость. Обозначить направление работы в будущем.
Возможно выделить несколько направлений дальнейшего развития методов эмпирического низкоразмерного моделирования. Необходимо также исследовать эмпирическое снижение размерности транспортных полей типа насыщенности фаз. В качестве преобразований, снижающих размерность, могут быть использованы нелинейные преобразования из сферы методов машинного обучения: необходимы сравнительные исследования применимости этих различных способов по аспектам качества, стабильности и доступности необходимых ресурсов на практике. В качестве перспективного аппарата инвариантного преобразования данных могут выступить так же нелинейные преобразования координат, что также требует отдельного изучения.
