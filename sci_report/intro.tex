\chapter*{Введение}                         % Заголовок
\addcontentsline{toc}{chapter}{Введение}    % Добавляем его в оглавление

\newcommand{\actuality}{}
\newcommand{\progress}{}
\newcommand{\aim}{{\textbf\aimTXT}}
\newcommand{\tasks}{\textbf{\tasksTXT}}
\newcommand{\novelty}{\textbf{\noveltyTXT}}
\newcommand{\influence}{\textbf{\influenceTXT}}
\newcommand{\methods}{\textbf{\methodsTXT}}
\newcommand{\defpositions}{\textbf{\defpositionsTXT}}
\newcommand{\reliability}{\textbf{\reliabilityTXT}}
\newcommand{\probation}{\textbf{\probationTXT}}
\newcommand{\contribution}{\textbf{\contributionTXT}}
\newcommand{\publications}{\textbf{\publicationsTXT}}


{\actuality} \todo{Актуальность работы}
% \LaTeX, введение в тему, обозначение места данной работы в
% мировых исследованиях и~т.\:п., можно использовать ссылки на~другие
% работы~\autocite{Gosele1999161,Lermontov}
% (если их~нет, то~в~автореферате
% автоматически пропадёт раздел <<Список литературы>>). Внимание! Ссылки
% на~другие работы в~разделе общей характеристики работы можно
% использовать только при использовании \verb!biblatex! (из-за технических
% ограничений \verb!bibtex8!. Это связано с тем, что одна
% и~та~же~характеристика используются и~в~тексте диссертации, и в
% автореферате. В~последнем, согласно ГОСТ, должен присутствовать список
% работ автора по~теме диссертации, а~\verb!bibtex8! не~умеет выводить в~одном
% файле два списка литературы).
% При использовании \verb!biblatex! возможно использование исключительно
% в~автореферате подстрочных ссылок
% для других работ командой \verb!\autocite!, а~также цитирование
% собственных работ командой \verb!\cite!. Для этого в~файле
% \verb!common/setup.tex! необходимо присвоить положительное значение
% счётчику \verb!\setcounter{usefootcite}{1}!.

% Для генерации содержимого титульного листа автореферата, диссертации
% и~презентации используются данные из файла \verb!common/data.tex!. Если,
% например, вы меняете название диссертации, то оно автоматически
% появится в~итоговых файлах после очередного запуска \LaTeX. Согласно
% ГОСТ 7.0.11-2011 <<5.1.1 Титульный лист является первой страницей
% диссертации, служит источником информации, необходимой для обработки и
% поиска документа>>. Наличие логотипа организации на~титульном листе
% упрощает обработку и~поиск, для этого разметите логотип вашей
% организации в папке images в~формате PDF (лучше найти его в векторном
% варианте, чтобы он хорошо смотрелся при печати) под именем
% \verb!logo.pdf!. Настроить размер изображения с логотипом можно
% в~соответствующих местах файлов \verb!title.tex!  отдельно для
% диссертации и автореферата. Если вам логотип не~нужен, то просто
% удалите файл с~логотипом.

% \ifsynopsis
% Этот абзац появляется только в~автореферате.
% Для формирования блоков, которые будут обрабатываться только в~автореферате,
% заведена проверка условия \verb!\!\verb!ifsynopsis!.
% Значение условия задаётся в~основном файле документа (\verb!synopsis.tex! для
% автореферата).
% \else
% Этот абзац появляется только в~диссертации.
% Через проверку условия \verb!\!\verb!ifsynopsis!, задаваемого в~основном файле
% документа (\verb!dissertation.tex! для диссертации), можно сделать новую
% команду, обеспечивающую появление цитаты в~диссертации, но~не~в~автореферате.
% \fi

{\progress} \todo{Степень разработанности}

{\aim} данной работы является разработка методологии, метрик качества и эффективных методов снижения размерности нелинейных динамических моделей вычислительной физики с управляющими воздействиями и неоднородными стационарными полями свойств среды.
В качестве методов снижения размерности рассматриваются методы анализа данных как альтернатива иным методам упрощения вычислений: снижение пространственно-временного разрешения численной модели, упрощение физической модели и т.п.
Одним из приложений разрабатываемой методологии являются обратные задачи моделирования подземных течений, пространство поиска которых может содержать координаты точечных источников и пространственное распределение фильтрационно-емкостных свойств.

Для~достижения поставленной цели необходимо было решить следующие {\tasks}:
\begin{enumerate}[beginpenalty=10000] % https://tex.stackexchange.com/a/476052/104425
  \item Проанализировать существующие исследования и разработки в контексте предмета данной работы.
  \item Определить основную проблематику, границы применимости существующих подходов.
  \item Сформулировать методологию и критерии оценки качества снижения размерности.
  \item Предложить алгоритмы, расширяющие границы применимости снижения размерности.
  \item Провести сравнительные численные эксперименты, проанализировать качество и производительность численного моделирования
  \item Сформулировать подходы к разработке дизайна прикладного программного обеспечения
\end{enumerate}

{\novelty}:
\begin{enumerate}[beginpenalty=10000] % https://tex.stackexchange.com/a/476052/104425
  \item Было выполнено оригинальное систематизирующее аналитическое исследование применимости научно-прикладных разработок в области данной работы применительно к моделированию пластовых течений в процессе разработки месторождений углеводородов.
  \item Разработан принцип повышения обобщающей способности линейного снижения размерности динамических моделей с точечными управляющими воздействиями и неоднородностями на основе инвариантных преобразований координат, элементов теории управления и теории оператора Купмана.
  \item В рамках предложенного принципа разработан набор методов снижения размерности для систем с точечным управлением и неоднородностями.
  \item Предложен с формальным обоснованием критерий сходимости моделей со сниженной размерностью, предназначенный для сопоставления со сходимостью соответствующей полноразмерной модели.
\end{enumerate}

{\influence} полученных результатов обеспечивается приближенностью рассматриваемых постановок задач к практическим задачам в сфере моделирования месторождений углеводородов. Учтены практические ограничения во времени, доступных вычислительных мощностях и данных. В представляемой работе рассматривается вопрос синергетической эффективности совокупности алгоритмов на примере открытой библиотеки для прямого и обратного нефтегазового моделирования, для которой было разработано расширение, воплощающее теоретические результаты данной работы.

\todo{\methods}

{\defpositions}
\begin{enumerate}[beginpenalty=10000] % https://tex.stackexchange.com/a/476052/104425
  \item \todo{это тезисы, которые никем ранее не были выдвинуты}
  \item \todo{Это своеобразные результаты научной деятельности, выводы, которые показывают, насколько полезно проведенное исследование и какова его ценность}
  \item \todo{https://disszakaz.ru/services/kandidatskaya-dissertatsiya /osnovnye-polozheniya-vynosimye-na-zashchitu-dissertatsii/}
  \item \todo{Def-positions.pdf}
\end{enumerate}

{\reliability} полученных результатов обеспечивается выбранной методологией исследования.
Разработанные подходы и алгоритмы тестировались с применением элементов теории численного эксперимента, статистической проверки гипотез, и принципа фальсифицируемости.
Ряд численных экспериментов проводились с использованием псевдо-случайных входных данных.
Кроме того, для верификации полученных результатов использовались данные, модели и алгоритмы других авторов.
Таким образом, воспроизводимость результатов оценивалась непосредственно в рамках исследования.
Структура и содержание данной работы обеспечивают возможность независимого воспроизведения численных моделей, разработанных методов и полученных результатов.

{\probation}
Основные результаты работы докладывались~на международной конференции \flqq{ECMOR}\frqq~\cite{Elizarev2022,Elizarev2020}, \todo{Международной конференции МФТИ}, в рамках международной конференции и выставки общества \flqq{SPE}\frqq~\cite{Elizarev_2019}, а также на семинарах кафедры моделирования и технологий разработки нефтяных месторождений МФТИ.

{\contribution} Автор выполнил исследование научно-технических достижений в области данной работы, интерпретировал и систематизировал их совокупность. Автор поставил и провёл необходимые теоретические и экспериментальные исследования, сформулировал проблематику и формальные постановки решаемых задач, подобрал инструменты программирования и программное обеспечение, проанализировал и оформил полученные результаты в виде научных публикаций и презентаций.

\ifnumequal{\value{bibliosel}}{0}
{%%% Встроенная реализация с загрузкой файла через движок bibtex8. (При желании, внутри можно использовать обычные ссылки, наподобие `\cite{vakbib1,vakbib2}`).
    {\publications} Основные результаты по теме диссертации изложены
    в~XX~печатных изданиях,
    X из которых изданы в журналах, рекомендованных ВАК,
    X "--- в тезисах докладов.
}%
{%%% Реализация пакетом biblatex через движок biber
    \begin{refsection}[bl-author, bl-registered]
        % Это refsection=1.
        % Процитированные здесь работы:
        %  * подсчитываются, для автоматического составления фразы "Основные результаты ..."
        %  * попадают в авторскую библиографию, при usefootcite==0 и стиле `\insertbiblioauthor` или `\insertbiblioauthorgrouped`
        %  * нумеруются там в зависимости от порядка команд `\printbibliography` в этом разделе.
        %  * при использовании `\insertbiblioauthorgrouped`, порядок команд `\printbibliography` в нём должен быть тем же (см. biblio/biblatex.tex)
        %
        % Невидимый библиографический список для подсчёта количества публикаций:
        \printbibliography[heading=nobibheading, section=1, env=countauthorvak,          keyword=biblioauthorvak]%
        \printbibliography[heading=nobibheading, section=1, env=countauthorwos,          keyword=biblioauthorwos]%
        \printbibliography[heading=nobibheading, section=1, env=countauthorscopus,       keyword=biblioauthorscopus]%
        \printbibliography[heading=nobibheading, section=1, env=countauthorconf,         keyword=biblioauthorconf]%
        \printbibliography[heading=nobibheading, section=1, env=countauthorother,        keyword=biblioauthorother]%
        \printbibliography[heading=nobibheading, section=1, env=countregistered,         keyword=biblioregistered]%
        \printbibliography[heading=nobibheading, section=1, env=countauthorpatent,       keyword=biblioauthorpatent]%
        \printbibliography[heading=nobibheading, section=1, env=countauthorprogram,      keyword=biblioauthorprogram]%
        \printbibliography[heading=nobibheading, section=1, env=countauthor,             keyword=biblioauthor]%
        \printbibliography[heading=nobibheading, section=1, env=countauthorvakscopuswos, filter=vakscopuswos]%
        \printbibliography[heading=nobibheading, section=1, env=countauthorscopuswos,    filter=scopuswos]%
        %
        \nocite{*}%
        %
        {\publications} Основные результаты по теме диссертации изложены в~\arabic{citeauthor}~печатных изданиях,
        \arabic{citeauthorvak} из которых изданы в журналах, рекомендованных ВАК\sloppy%
        \ifnum \value{citeauthorscopuswos}>0%
            , \arabic{citeauthorscopuswos} "--- в~периодических научных журналах, индексируемых Web of~Science и Scopus\sloppy%
        \fi%
        \ifnum \value{citeauthorconf}>0%
            , \arabic{citeauthorconf} "--- в~тезисах докладов.
        \else%
            .
        \fi%
        \ifnum \value{citeregistered}=1%
            \ifnum \value{citeauthorpatent}=1%
                Зарегистрирован \arabic{citeauthorpatent} патент.
            \fi%
            \ifnum \value{citeauthorprogram}=1%
                Зарегистрирована \arabic{citeauthorprogram} программа для ЭВМ.
            \fi%
        \fi%
        \ifnum \value{citeregistered}>1%
            Зарегистрированы\ %
            \ifnum \value{citeauthorpatent}>0%
            \formbytotal{citeauthorpatent}{патент}{}{а}{}\sloppy%
            \ifnum \value{citeauthorprogram}=0 . \else \ и~\fi%
            \fi%
            \ifnum \value{citeauthorprogram}>0%
            \formbytotal{citeauthorprogram}{программ}{а}{ы}{} для ЭВМ.
            \fi%
        \fi%
        % К публикациям, в которых излагаются основные научные результаты диссертации на соискание учёной
        % степени, в рецензируемых изданиях приравниваются патенты на изобретения, патенты (свидетельства) на
        % полезную модель, патенты на промышленный образец, патенты на селекционные достижения, свидетельства
        % на программу для электронных вычислительных машин, базу данных, топологию интегральных микросхем,
        % зарегистрированные в установленном порядке.(в ред. Постановления Правительства РФ от 21.04.2016 N 335)
    \end{refsection}%
    \begin{refsection}[bl-author, bl-registered]
        % Это refsection=2.
        % Процитированные здесь работы:
        %  * попадают в авторскую библиографию, при usefootcite==0 и стиле `\insertbiblioauthorimportant`.
        %  * ни на что не влияют в противном случае
        \nocite{vakbib2}%vak
        \nocite{patbib1}%patent
        \nocite{progbib1}%program
        \nocite{bib1}%other
        \nocite{confbib1}%conf
    \end{refsection}%
        %
        % Всё, что вне этих двух refsection, это refsection=0,
        %  * для диссертации - это нормальные ссылки, попадающие в обычную библиографию
        %  * для автореферата:
        %     * при usefootcite==0, ссылка корректно сработает только для источника из `external.bib`. Для своих работ --- напечатает "[0]" (и даже Warning не вылезет).
        %     * при usefootcite==1, ссылка сработает нормально. В авторской библиографии будут только процитированные в refsection=0 работы.
}
 % Характеристика работы по структуре во введении и в автореферате не отличается (ГОСТ Р 7.0.11, пункты 5.3.1 и 9.2.1), потому её загружаем из одного и того же внешнего файла, предварительно задав форму выделения некоторым параметрам

\textbf{Объем и структура работы.} Диссертация состоит из~введения,
\formbytotal{totalchapter}{глав}{ы}{}{},
заключения и
\formbytotal{totalappendix}{приложен}{ия}{ий}{}.
%% на случай ошибок оставляю исходный кусок на месте, закомментированным
%Полный объём диссертации составляет  \ref*{TotPages}~страницу
%с~\totalfigures{}~рисунками и~\totaltables{}~таблицами. Список литературы
%содержит \total{citenum}~наименований.
%
Полный объём диссертации составляет
\formbytotal{TotPages}{страниц}{у}{ы}{}, включая
\formbytotal{totalcount@figure}{рисун}{ок}{ка}{ков} и
\formbytotal{totalcount@table}{таблиц}{у}{ы}{}.
Список литературы содержит
\formbytotal{citenum}{наименован}{ие}{ия}{ий}.
